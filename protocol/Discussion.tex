\newpage
\section{Discussion}
The data showed a detection threshold of about 0.06 s in the 4AFC task, with results varying between individuals and a dependency on spatial frequency: The Gabor patches with the lower spatial frequency (2 cpd) had a lower detection threshold than the patches with higher spatial frequency (8 cpd) and were thus perceived more easily. In the comparative visual search task, on the population level, only response and processing time were significantly influenced by the different spatial frequencies, the varying delay had no significant effect on the subjects behaviour. On the individual level, the strategy space spanned by processing time and number of switches fitted a power function. The strategy index revealed a slight shift towards memorization in many individuals for both the higher spatial frequency as well as the 2 s delay condition. No correlation of mean detection threshold in the 4AFC with the strategy index could be found. The data sampled in the contrast sensitivity experiment found a detection threshold at a contrast level of about 0.017 in the 10 cpd condition, in the 20 cpd condition the detection threshold was not within the range used in this experiment.
\newline
Overall, the 4AFC experiment conducted as part of this project was able to replicate the effect expected. Higher spatial frequencies take longer to be processed, presumably because of the different gain of the M and P channel (Murray and Plainis 2003), leading to higher detection thresholds regarding the time the stimulus is presented necessary for reliable detection. This effect would also account for the difference in detection threshold for the contrast levels tested in the third experiment, on contrast sensitivity. 
\newline 
As to why some participants were unable to reliably detect the stimuli presented within the given range of periods of display, while one subject even exceeded the range towards 0 s, we could not find any suggestions in the literature but think it would be worth going over the stimulus presentation style again and try to find strategies improving or impeding performance. 
\newline
\newline
The subjects' performance in the comparative visual search task was overall stable and could show trends in some of the effects expected. The constant error rate and trial duration for most of the participants over the number of trials shows that the ongoing task did not alter the performance or strategy. The population level analysis could only show a significant influence 
 of spatial frequency on the processing and response time, suggesting that the subjects mostly just accepted the higher costs for perception and thus memorization and took more time in these trials, keeping consistent with the number switches and the error rate. The Wilcoxon signed-rank tests were non-significant for the two delay conditions, maybe because the delay was too short to substantially increase the acquisition costs, or because the subjects' strategy was laid out on minimal acquisition from the beginning on. The former reason would not be in line with the findings of an effect by Hardiess and Mallot (2015), who used 1.5 s as delay. We therefore rather hypothesize that the delay used was actually too long: First of all, it seems a little bit unrealistic, as no saccade would take so long, which might have influenced the participants' behaviour. But more importantly, a delay of that length does not only put costs on acquisition, but on memorization too, as information needs to be maintained over this period. This eventually equal rise in costs eue to the cross-interaction might be the reason explaining the unchanged, or even reversed strategy of some participants. 
 \newline
 \newline
 Analysis of the individual strategies revealed an alignment of individuals along a power-fit function in the strategy space of processing time over number of switches. Congruent with the expectation, subjects with higher numbers of switches had a shorter processing time. This directly demonstrates an example of the trade-off between memorization and acquisition in this task. In this individual analysis, other than in the population, it can also be seen, that the longer delay shifts the strategies towards memorization. This is not only visible in the shift of the power-fit function, but very prominently for three individuals in the shift of the strategy index. The influence of the higher spatial frequency on memorization costs and thus strategy seen in the overall population is however, although very consistent, not large. We generally think, that more pronounced differences in acquisition and memorization costs might solve this problem and lead to significant changes in strategy. In order to overcome the effect of parallel rise in acquisition and memorization costs with longer delays it might help to increase the energy costs of the acquisition instead. 
 \newline The fact that no correlation between mean detection threshold and strategy index was found is not surprising, as many other factors such as concentration level or personality may have influenced either experiment.
 \newline
 \newline
 Finally, our analysis of the experiment on the relation of contrast sensitivity and spatial frequency in sinusoidally grated targets such as our Gabor patches showed an influence of contrast level on perceivability of a stimulus as well as an influence of the spatial frequency of the stimulus on the contrast sensitivity as described by Campbell and Maffei (1974). However, the 20 cpd stimuli were apparently so difficult to perceive that within the contrast range given no subject could perform significantly above chance. The thresholds for 10 cpd and interpolated from the results of the subjects for 20 cpd however lie on the  spatial frequency contrast
sensitivity function (also referred to as Modulation Transfer Function; Arden (1978)) found by Campbell and Maffei (1974) and later Campbell (1983) at the respective spatial frequencies.